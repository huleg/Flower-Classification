\documentclass[11pt, a4paper]{report}
\usepackage{fontspec}
\setmainfont{Arial}

\begin{document}

\chapter{Introduction}

\section{Definition of the problem} 

\section{Why problem is worth solving, challenges which occur during this type of classification}

\section{Achievements}







\chapter{Literature Review}

couple of pages long 3-5

Gives reader knowledge of what people have done before on this topic. 

Have people writen apps like this before (there are for dogs, leaves) - look up these papers and compare them.

Look over previous papers on flower classification 

CNNs, where they came from, who invented them, imagenet challenge


Describe data - how many classes, how many images, give examples of flowers


\chapter{Classification}

\section{Overview of classification pipeline}
Photo of a flower is taken by the user. CNN converts the photo into a characteristic vector. SVM classifiers used to determine the species of flower. 

\section{Convolutional neural network inspired feature extraction}
Used to generate characteristic vector from an image. Treated as a black box in this project. Feature extraction (not end to end). Borrow graphs from Andrea. Look up his papers on his network. How the network was trained - imagenet. Look up the data it was trained on. 

\section{Support Vector Machine for image classification}

\subsection{How SVMs work}

\subsection{Training and testing SVMs}


\section{Experiments - this should be half the content. 7 pages, half should be experiments}

\subsection{SVM accuracy}

\subsection{Improving SVM accuracy}

\subsubsection{Mirroring}

\subsubsection{Sampling (Jittering?)}

\subsection{How SVMs are used in the classification pipeline}

\section{REMOVE: MatLab classification script}

\subsection{Explication of classification script}
Script uses the CNN to generate a characteristic vector from an inputted photo. We test that vector against the 102 weight vectors found during training to produce a classification prediction.





\chapter{Client architecture join chapter with Server. }

TODO: don't group sections by activities (this is an Android specific way of doing things Keep client section quite short 3 pages)


\section{Overview of client architecture}
Android application consists of three activities, or screens with which the user interacts. The Main Activity allows the user to take a photo or choose a photo from gallery to upload. The photo is uploaded to the server, which classifies the flower, and returns the result to the Results Activity. The results activity displays the eight top classification results, and allows the user to click on each, and find out more information in the Detail Activity. 

\section{Main Activity}

Main Activity consists of two buttons, which launch the camera and gallery image picker respectively.

\subsection{Camera intent}

\subsection{Gallery image picker intent}

\section{Results Activity}

Description of code used in the Results Activity

\subsection{Uploading the photo}

\subsubsection{AsyncTask (keeping slow processes such as web connections off the UI thread)}

\subsubsection{Connecting to the server}

\subsection{Loading classification results}

\subsubsection{Parsing JSON Array}

\subsubsection{ListView}

\subsubsection{Picasso image downloader library}

\section{Detail Activity}

Description of code used in the Detail Activity

\subsection{TODO after Detail Activity implemented}








\chapter{Server architecture}

\section{Overview of server architecture}

\section{Flask server}

\subsection{How server accepts the photo}

\subsection{Error catching}

\subsubsection{Safe filenames}

\subsubsection{Allowed filetypes}

\section{Connection between servers}

\section{Backend server}

\subsection{Description of how server works}

\subsection{mlwrap}








\chapter{Conclusions and Future Work: mirror on intro. Intro discusses challenges, conclusions describe how challenges were minimised. Were goals achieved}

\section{Conclusions}

\section{Porting algorithm to Android}



\end{document}
